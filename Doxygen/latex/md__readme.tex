Super-\/simple monitoring program.

{\ttfamily mon} spawned from the needlessly complex frustration that tools like \href{http://mmonit.com/monit/}{\tt monit} provide, with their awkward D\+S\+Ls and setup. {\ttfamily mon} is written in C, uses less than 400kb of memory, and is incredibly simple to set up.

\subsection*{Installation}


\begin{DoxyCode}
$ make install
\end{DoxyCode}


Too lazy to clone?\+:


\begin{DoxyCode}
$ (mkdir /tmp/mon && cd /tmp/mon && curl -L# https://github.com/tj/mon/archive/master.tar.gz | tar zx
       --strip 1 && make install && rm -rf /tmp/mon)
\end{DoxyCode}


\subsection*{Usage}


\begin{DoxyCode}
Usage: mon [options] <command>

Options:

  -V, --version                 output program version
  -h, --help                    output help information
  -l, --log <path>              specify logfile [mon.log]
  -s, --sleep <sec>             sleep seconds before re-executing [1]
  -S, --status                  check status of --pidfile
  -p, --pidfile <path>          write pid to <path>
  -m, --mon-pidfile <path>      write mon(1) pid to <path>
  -P, --prefix <str>            add a log prefix
  -d, --daemonize               daemonize the program
  -a, --attempts <n>            retry attempts within 60 seconds [10]
  -R, --on-restart <cmd>        execute <cmd> on restarts
  -E, --on-error <cmd>          execute <cmd> on error
  -r, --memory                  write memory usage to logfile
\end{DoxyCode}


\subsection*{Example}

The most simple use of {\ttfamily mon(1)} is to simply keep a command running\+:


\begin{DoxyCode}
$ mon ./myprogram
mon : pid 50395
mon : child 50396
mon : sh -c "./example/program.sh"
one
two
three
\end{DoxyCode}


You may daemonize mon and disassociate from the term with {\ttfamily -\/d}\+:


\begin{DoxyCode}
$ mon ./myprogram -d
mon : pid 50413
\end{DoxyCode}


\subsection*{Failure alerts}

{\ttfamily mon(1)} will continue to attempt restarting your program unless the maximum number of {\ttfamily -\/-\/attempts} has been exceeded within 60 seconds. Each time a restart is performed the {\ttfamily -\/-\/on-\/restart} command is executed, and when {\ttfamily mon(1)} finally bails the {\ttfamily -\/-\/on-\/error} command is then executed before mon itself exits and gives up.

For example the following will echo \char`\"{}hey\char`\"{} three times before mon realizes that the program is unstable, since it\textquotesingle{}s exiting immediately, thus finally invoking {\ttfamily ./email.sh}, or any other script you like.

\`{}\`{}\`{}js mon \char`\"{}echo hey\char`\"{} --attempts 3 --on-\/error ./email.sh mon \+: child 48386 mon \+: sh -\/c \char`\"{}echo hey\char`\"{} hey mon \+: last restart less than one second ago mon \+: 3 attempts remaining mon \+: child 48387 mon \+: sh -\/c \char`\"{}echo hey\char`\"{} hey mon \+: last restart less than one second ago mon \+: 2 attempts remaining mon \+: child 48388 mon \+: sh -\/c \char`\"{}echo hey\char`\"{} hey mon \+: last restart less than one second ago mon \+: 1 attempts remaining mon \+: 3 restarts within less than one second, bailing mon \+: on error {\ttfamily sh test.\+sh} emailed failure notice to \href{mailto:tobi@ferret-land.com}{\tt tobi@ferret-\/land.\+com} mon \+: bye \+:) 
\begin{DoxyCode}
  \_\_NOTE\_\_: The process id is passed as an argument to both `--on-error` and `--on-restart` scripts.

## Managing several mon(1) processes

  `mon(1)` is designed to monitor a single program only, this means a few things,
  firstly that a single `mon(1)` may crash and it will not influence other programs,
  secondly that the "configuration" for `mon(1)` is simply a shell script,
  no need for funky weird inflexible DSLs.

```bash
#!/usr/bin/env bash

pids="/var/run"
app="/www/example.com"

mon -d redis-server -p $pids/redis.pid
mon -d "node $app/app" -p $pids/app-0.pid
mon -d "node $app/jobs" -p $pids/jobs-0.pid
mon -d "node $app/jobs" -p $pids/jobs-1.pid
mon -d "node $app/jobs" -p $pids/jobs-2.pid
mon -d "node $app/image" -p $pids/image-0.pid
mon -d "node $app/image" -p $pids/image-1.pid
mon -d "node $app/image-broker" -p $pids/image-broker.pid
\end{DoxyCode}


I highly recommend checking out jgallen23\textquotesingle{}s \href{https://github.com/jgallen23/mongroup}{\tt mongroup(1)}, which provides a great interface for managing any number of {\ttfamily mon(1)} instances.

\subsection*{Logs}

By default {\ttfamily mon(1)} logs to stdio, however when daemonized it will default to writing a log file named {\ttfamily ./mon.log}. If you have several instances you may wish to {\ttfamily -\/-\/prefix} the log lines, or specify separate files.

\subsection*{Signals}


\begin{DoxyItemize}
\item {\bfseries S\+I\+G\+Q\+U\+IT} graceful shutdown
\item {\bfseries S\+I\+G\+T\+E\+RM} graceful shutdown
\end{DoxyItemize}

\subsection*{Links}

Tools built with {\ttfamily mon(1)}\+:


\begin{DoxyItemize}
\item \href{https://github.com/jgallen23/mongroup}{\tt mongroup(1)} -\/ monitor a group of processes (shell script)
\item \href{https://github.com/visionmedia/node-mongroup}{\tt node-\/mongroup} -\/ node implementation of mongroup(1)
\end{DoxyItemize}

\section*{License}

M\+IT

\section*{Build Status}

\href{http://travis-ci.org/visionmedia/mon}{\tt } 